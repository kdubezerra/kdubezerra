\documentclass[times, 10pt]{article} 
%\documentclass[times, 10pt,twocolumn]{article} 
%\usepackage[noend]{distribalgo}
\usepackage{algorithm}
%\usepackage{times}
\usepackage{amsmath}
\usepackage{graphicx}
\usepackage{color}
\usepackage{amssymb}
\usepackage[draft]{fixme}

\begin{document}

\newcommand{\mv}[1]{\ensuremath{\operatorname{\mathit{#1}}}}
\definecolor{dark}{gray}{.6}
\newcommand{\bc}[1]{\textcolor{dark}{#1}}
\newtheorem{lems}{Lemma}
\newtheorem{props}{Proposition}
\newtheorem{thms}{Theorem}
\newtheorem{defs}{Definition}
\newtheorem{obs}{Observation}

\newcommand{\code}[1]{\texttt{\small{\textbf{#1}}}}

\title{Practical consistency management for geographically distributed MMOG servers}

%\author{
%xxx\\
%University \\ Country\\
%\and
%xxx\\
%University \\ Country\\
%\and
% ...
%}

\maketitle

\begin{abstract}


\end{abstract}

\section{Introduction}
\label{sec:intro}


\section{System model and definitions}
\label{sec:model}

We assume a system composed of nodes, divided into \emph{players} and \emph{servers}, distributed over a geographical area. Nodes may fail by crashing and subsequently recover, but do not experience arbitrary behavior (i.e., no Byzantine failures). Communication is done by message passing, through the primitives \emph{send}$(p,m)$ and \emph{receive}$(m)$, where $p$ is the addressee of message $m$. Messages can be lost but not corrupted. If a message is repeatedly resubmitted to a \emph{correct node} (defined below), it is eventually received.

Our protocols ensure safety under both asynchronous and synchronous execution periods. The FLP impossibility result~\cite{FLP85} states that under asynchronous assumptions consensus cannot be both safe and live. We thus assume that the system is initially asynchronous and eventually becomes synchronous. The time when the system becomes synchronous is called the \emph{Global Stabilization Time (GST)}~\cite{DLS88}, and it is unknown to the nodes.
Before GST, there are no bounds on the time it takes for messages to be transmitted and actions to be executed. After GST, such bounds exist but are unknown. After GST nodes are either \emph{correct} or \emph{faulty}. A correct node is operational ``forever" and can reliably exchange messages with other correct nodes. This assumption is only needed to prove liveness properties about the system. In practice, ``forever" means long enough for one instance of consensus to terminate.

A game is composed of a set of objects. The game state is defined by the individual states of each one of its objects. We assume that the game objects are partitioned among different servers. Since objects have a location in the game, one way to perform this partitioning is by zoning the virtual world of the game. Each partition consists of a set of objects of the game and the server responsible for them is their \emph{coordinator}.

Each player sends its commands to one of the servers\footnote{We consider that each server is replicated and that each player may send his command to one of the replicas, instead of the main server, if that provides a lower delay between the issuing of a command and its delivery.} -- in the case of avatar based virtual environments, and as an avatar is usually also an object, the server to which a player is connected is his avatar's coordinator. A command $C = \{ c_1, c_2, ... \}$ is composed of one or more subcommands, one subcommand per object it affects. We refer to the set of objects affected by command $C$ as $obj(C)$. Also, we refer to the objects coordinated by a server $S$ as $obj(S)$. Finally, we define $obj(C,S)$ = \{$o$ : $o \in obj(C)$ and $o \in obj(S)$\}.

%Each object has its own command stream. When a command affects more than one object, it must be inserted in the command stream of all the objects it affects.

Our consistency criterion is ``eventual linearizability". (I'm not sure this is indeed what we want and how to define it, but we do need some consistency criterion...)


%In order to guarantee that commands are executed properly by servers, theywe define the following order requirement.
% 
%Let $G =(V,E)$ be a graph where the set $V$ of vertices contains all game commands and $E$ contains edge $C_i \rightarrow C_j$ if subcommand $c_x \in C_i$ is executed before subcommand $c_y \in C_j$. The order requirement states that $G$ is acyclic. In other words, if two commands affect common objects, their subcommands must be executed in the same order.
%
%Our goal is to design ordering protocols with low latency. Two aspects affect latency: (a)~the number of communication steps needed to ensure that subcommands can be properly executed in the absence of contention and (b)~possible slowdowns due to contention. We illustrate these aspects in the following section.
%
%To each partition, a group of servers is assigned, and they are responsible for storing and managing the state of those objects. One of them is a coordinator (or region manager, if one considers partitions as regions) and the rest of the group is formed by replicas, whose purpose is to provide fault tolerance.



\section{Baseline protocol}

To ensure reliability despite server crashes, each server is replicated using state-machine replication, implemented with Paxos. Each player may send its commands to any of his server's replicas -- which then forwards it to the server. Each command is assigned a timestamp and executed against objects in timestamp order. We implement this by assigning each object a logical clock. If the timestamp of a command $C$ is smaller than the logical clock of the object it affects, $C$ is not executed against the object; otherwise, $C$ is executed and its timestamp is used to set the logical clock of the objects if affects. Therefore, the challenge is how to assign timestamps to commands such that consistency is not violated and commands are not discarded due to stale timestamp values. There are three cases to consider:


%In brief, our strategy is to combine state-machine replication (implemented with Paxos), used to handle object replicas, with Skeen's algorithm, used to select timestamps for commands that affect multiple objects.

\begin{itemize}
\item Command $C$ affects the state of only one game object.

In this case, upon receiving $C$, the server forwards $C$ to the object coordinator, which, in turn, locks\footnote{Locking an object here means that no two command proposals can be done in parallel for that object. If, while trying to lock an object, it was already locked, the \code{\footnotesize{lock()}} call returns a value informing that it failed to lock.} the object and starts an instance of Paxos in its replication group to assign a timestamp to $C$. When that run of Paxos finishes, a timestamp $TS$ is assigned to $C$, the object is unlocked and each server of the group delivers $C$ and executes it as soon as $C$'s timestamp is the smallest one affecting the referred object.

\item Command $C$ affects two or more objects, all of which are managed by the same server group.

The server that received $C$ then forwards it to the group coordinator. In the command, it is informed which objects are affected by it. With this information, then, the coordinator locks all the involved objects, picks the highest value ($TS$) among their logical clocks and assigns it as a timestamp for $C$. After that, an instance of Paxos is run in order to store the command and its timestamp in the memory of replicas. Upon reaching consensus, the logical clocks of all the affected objects are, then, set to a value higher than $TS$ and they are all unlocked.

\item Command $C$ affects two or more objects and at least two of them are not managed by the same server group.

\code{(we have to check whether this really works\ldots)}

We rely on a variation of Skeen's algorithm to implement this case. The server that receives $C$ forwards $C$ to all the coordinators of the objects affected by $C$. Each coordinator $S$ then locks all objects in $obj(C,S)$ and proposes a timestamp for $C$. ``To propose" here means to achieve consensus (with Paxos, for example) in the group regarding $C$'s timestamp proposed by that group. The value proposed by a group coordinator when it initiates the consensus instance in its group corresponds to the highest value among the logical clocks of the objects in $obj(C,S)$.

After a group reaches consensus about its timestamp proposal for $C$, it sends it to the other groups. When a group with a coordinator $S$ possesses all the timestamp proposals, it picks the highest among them ($TS$), assigns it as a timestamp to $C$, and unlocks all objects in $obj(C,S)$. For each group whose proposal was not the highest one, the logical clocks of its object(s) affected by $C$ must be increased to a value higher than $TS$, which is done by means of consensus in the group.

\end{itemize}




%A common mechanism to handle replication is the state-machine replication approach~\cite{Lam78}. This mechanism ensures that for any given object, commands are executed in the same order across all object replicas. To guarantee the command order property, we can use Skeen's multicast algorithm, which has been proved to be optimum in number of steps.
%
%Skeen+Paxos
%
%, and thus it also has its own logical clock and its timestamps are assigned independently from those of other objects, unless a command affects two or more objects. A server executes commands in timestamp order, that is, command C with timestamp tmp(C) is only executed against object O if tmp(O) < tmp(C). After the execution of C, O�s timestamp is updated to C�s.
%
%To ensure that C will be executed by all objects concerned by C, its timestamp tmp(C) will be set to the highest value among the logical clocks of all the objects it was addressed to. After the execution of C, the logical clock of each object is set to a value higher than tmp(C).
%
%
%
%Here, it is considered that to each command is assigned a timestamp, according to Lamport's logical clocks. 
%


\section{Optimistic protocol}

By understanding the abstract description of the baseline protocol of the previous section, it is easy to realize that, even when ordering commands relevant to only one group, a significant number of communication steps is required: player to replica; replica to coordinator; consensus (two communication steps in the case of Paxos, when having the coordinator as the leader, using ballot reservation, each acceptor sending phase 2b messages to all other acceptors and assuming a tolerable number of failures) and, once consensus is achieved, each replica can put the command in the delivery queue and send it to each player. That sums up to five communication steps. On a geographically distributed scenario, this number of communication steps may be prohibitive for the ``playability'' of a vast number of online games.

In order to mitigate this problem, we propose a protocol for \emph{optimistic ordering and delivery} of the commands, which is supposed to run in parallel with the protocol described in the previous section, which from now on will be called \emph{conservative ordering and delivery}. The optimistic order and delivery are correct if they include all the messages delivered with the conservative one and if they are delivered in the same order. Also, each object has an optimistic state and a conservative state. When a command is conservatively delivered to an object, its conservative state is updated, and this state is certainly consistent -- given the assumptions we have made about the system. Likewise, when a command is optimistically delivered, its optimistic state is updated.

The optimistic ordering and delivery is supposed to be much faster than the conservative one. Assuming that it will be, and if the optimistic delivery order is always correct -- which means that it correctly predicted the conservative delivery order --, then the objects will always have a valid state within much fewer communication steps, counting from when each command $C$ was issued to when it is delivered and applied to $obj(C)$.

The basic idea of the optimistic ordering is the following: assuming that the nodes have a synchronized clock\footnote{We don't require here perfectly synchronized clocks, as the optimistic protocol tolerates mistakes by its very definition. We only need clocks which are synchronized enought, so that our delivery order prediction succeeds and matches the conservative delivery order.}, whenever a server (either coordinator or replica) receives a command $C$ from a client, it immediately applies a provisory (optimistic) timestamp $pTS$ for it, which consists simply of the current value of the server's wallclock, $now$. Therefore, $pTS=now$. A wait window of length $w(S)$ is considered, where $w(S)$ is defined as the highest estimated communication delay between the server $S$ and any of the other servers of its group.

After applying the provisory timestamp to $C$, the server immediately forwards it to the other servers of that group, and puts $C$ on a list, where it stays until $now>C.pTS+w(S)$, which means that $C$ has been in that list for a time longer than the defined wait window. In the meantime, other commands, sent from other clients and forwarded by other servers, may have been received and also inserted in that list, always sorting by the provisory timestamp of these commands. If $w(S)$ has been correctly estimated as the highest communication delay between the server and each of the other servers, and no message was lost, than all the commands that were supposed to be delivered before $C$ have necessarily been received already. If the same occurred for all the servers, then all of them have received all the commands, and can deliver them in the same order of $pTS$.

However, even if the optimistic order is the same for all the servers, it won't be valid if the conservative order is different from it. For that reason, before submitting a command to be conservatively ordered, the coordinator $S_c$ also waits for $w(S_c)$, which is the maximum delay between $S_c$ and any of its replicas. Assuming that the clocks are sufficiently synchronized and that the communication delays have been correctly estimated, then after the time $w(S_c)$ has elapsed, the coordinator will have received all the commands forwarded by any of its replicas (until the instant $t:t=now-w(S_c)$), and it will be able to conservatively order them in the same order of the optimistic delivery, assuming that the final order is the same as the order in which the messages were submitted by the coordinator (FIFO). Therefore, given the optimistic assumptions we have made, the conservative delivery order (based on the order in which the coordinator FIFO-broadcast the commands to the group) will be the same as the optimistic delivery order (based on the provisory timestamp assigned by the replicas).

\subsection{Recovering from mistakes}

Unfortunately, even with a very good delay estimation (e.g. on an environment with a low jitter), there is absolutely no guarantee that the optimistic delivery order will match the conservative one. When it doesn't, it is considered a \emph{mistake}. Every mistake of the optimistic algorithm -- either a lost command message, or an out-of-order delivery -- will cause a rollback of the optimistic state of the objects and re-execution of some of the optimistically delivered commands.

To perform that, we consider that each object has an optimistic delivery queue, $Q_{opt}$. Whenever a command is optimistically delivered, the optimistic state is updated and the command is pushed in the back of $Q_{opt}$. Whenever a command $C_c$ is conservatively delivered, it updates the conservative state of each object in $obj(C_c)$ and, for each one of them, the algorithm checks whether it is the first command in $Q_{opt}$. If it is, $C_c$ is simply removed from $Q_{opt}$ and the execution continues. If it isn't, it means that $C_c$ was either optimistically delivered out of order, or it was simply never optmistically delivered. It then checks whether $Q_{opt}$ contains $C_c$. If it does, it means the command was optimistically deliverd out of order, and it is removed from the list -- if $Q_{opt}$ doesn't contain $C_c$, it was probably lost\footnote{Also, when $C_c$ is delivered, but it is not in $Q_{opt}$, the remaining possibility is the very unlikely case where the conservative delivery happened before the optimistic one. To handle this case, $C_c$ is stored in a list of possibly delayed optimistic delivery and, if it is ever optimistically delivered, the algorithm will know that it should only discard that command, instead of updating the optimistic state.}. Then, the optimistic state is overwritten with the conservative one and, from that state, all the remaining comands in $Q_{opt}$ are re-executed, leading to a new optimistic state for that object.


%\bibliographystyle{latex8}
%\bibliography{main}

\end{document}

