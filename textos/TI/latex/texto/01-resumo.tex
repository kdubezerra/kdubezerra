% resumo na l�ngua do documento
\begin{abstract}

Este documento � um exemplo de como formatar documentos para o
Instituto de Inform�tica da UFRGS usando as classes \LaTeX\
disponibilizadas pelo UTUG\@. Ao mesmo tempo, pode servir de consulta
para comandos mais gen�ricos. \emph{O texto do resumo n�o deve
conter mais do que 500 palavras.}
\end{abstract}

% resumo na outra l�ngua
% como parametros devem ser passados o titulo e as palavras-chave
% na outra l�ngua, separadas por v�rgulas
\begin{englishabstract}{Using \LaTeX\ to Prepare Documents at II/UFRGS}{Electronic document preparation, \LaTeX, ABNT, UFRGS}
This document is an example on how to prepare documents at II/UFRGS
using the \LaTeX\ classes provided by the UTUG\@. At the same time, it
may serve as a guide for general-purpose commands. \emph{The text in
the abstract should not contain more than 500~words.}
\end{englishabstract}

