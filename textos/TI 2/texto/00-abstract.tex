% resumo na l�ngua do documento
\begin{abstract}

Electronic games have become undoubtedly popular in the last few decades. This popularity gained an even higher raise when the multiplayer feature was added -- multiple players could play with, or against, each other in the same match. The latest advance, however has been the massively multiplayer online games -- or MMOGs, in short --, which allows the simultaneous participation of tens of thousands of players in the same game. However, to cope with such a demanding application, a powerful infrastructure is required: be it a powerful server-cluster or a geographically distributed collection of low-cost volunteer server nodes, there must be some way to distribute the load imposed by the game on the many computers involved in serving it. This work presents some interesting papers related to load balancing techniques used in massively multiplayer online games, commenting each one of them about its relevance. In the end, it is presented an idea of a possible improvement for load balancing by using a generic binary space partition (BSP) tree.

\end{abstract}

%% resumo na outra l�ngua
%% como parametros devem ser passados o titulo e as palavras-chave
%% na outra l�ngua, separadas por v�rgulas
%\begin{englishabstract}{Using distributed system as massively multiplayer game servers}
%
%This work reviews the state of the art in what concerns with the utilization of a geographically distributed system as a massively multiplayer game server. It has been researched works which employ the most recent techniques for this purpose, as well as their aspects concerning consistency, security, scalability e bandwidth usage. Besides, comments are made about the strong and weak points of each work.
%
%\end{englishabstract}

