\chapter{Final remarks}

The works presented here represent some of what is the newest techniques in terms of load distribution in multiserver systems for massively multiplayer online games. They all take in account the most important resource in a massively multiplayer game server: network bandwith. Although processing power used is also a resource to be used by the game servers, certainly the greatest concern is with the bandwidth occupation. To reach a saturation level where a game server CPU is overloaded, it is required a number of players much higher than that which would saturate the network connection of an average desktop personal computer.

Besides, comments were made -- that were disposed throughout the text and in the sections of "work analysis"{} -- for each of the presented works, showing what looks good and what appears to be bad in each one of them. While there are some weaknesses in the papers presented, one can make use of several ideas proposed by them, in order to create a new, more refined approach to balance the load in a massively multiplayer games multiserver system.


\section{Future work}

Considering what has been shown in the works presented, a possible increment would be the use of a generic BSP tree. In that data structure -- which is a more general case of the KD-tree -- each node of the tree contains a linear equation that represents a straight line in the plane. This line, then, divides the region represented by that tree node in two sub-regions, each one represented by other tree nodes. These tree nodes, therefore, are children of the first one.

By using such a more generic space partitioning data structure, such as the described BSP tree, the load balancing algorithm may achieve an ever better distribution than that made with the KD-tree and the other techniques presented in this work. This structure, however, would cause a significantly higher rebalancing complexity, as each tree node defines an arbitrary line, represented as a linear equation, instead of a simple $x$ or $y$ coordinate.

Another possible improvement would be to create a tree whose number of leaf nodes is higher than the number of servers in the multiserver system. This way, it could be designated multiple leaf nodes -- that is, multiple regions -- to each server. With such an approach, an even finer and more fairer load balacing. However, to allow for the multiple node assignment for each server, a different criterion for the region distribution must be devised.