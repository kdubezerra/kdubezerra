\chapter{Considera��es finais}

Os trabalhos aqui apresentados representam algumas do que h� de mais novo em termos de t�cnicas de distribui��o de servidor para jogos maci�amente multijogador. Um ponto que todas elas t�m em comum � o fato de que sup�e-se que este sistema servidor distribu�do est� instalado em uma rede local ou, no m�nimo numa rede em que o atraso � negligenci�vel e a largura de banda � abundante.

No entanto, se o objetivo de um trabalho futuro � dar suporte distribu�do a servi�o de jogos multijogador em uma rede com recursos limitados e vol�teis, como a Internet, deve-se reavaliar e analisar a aplicabilidade de cada uma das t�cnicas aqui revistas. Pode-se, tamb�m, buscar formar uma am�lgama das abordagens vistas, de forma a maximizar a efici�ncia de uso dos recursos dispon�veis, permitindo que haja um bom jogo mesmo em uma situa��o em que o atraso da comunica��o e a largura de banda estejam muito abaixo daquelas encontradas em redes locais.

Foram feitos coment�rios - que foram dispostos ao longo do texto e nas se��es de "avalia��o do trabalho"{} - para cada um dos trabalhos citados, mostrando o que aparenta ser bom e o que aparenta ser ruim na proposta de cada um dos autores. Embora existam alguns pontos imperfeitos nos trabalhos apresentados, pode-se aproveitar diversas id�ias por eles propostas, com o fim de criar uma nova abordagem mais refinada para dar suporte distribu�do a jogos maci�amente multijogador.