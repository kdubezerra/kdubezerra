%% LaTeX Beamer presentation template (requires beamer package)
%% see http://latex-beamer.sourceforge.net/
%% idea contributed by H. Turgut Uyar
%% template based on a template by Till Tantau
%% this template is still evolving - it might differ in future releases!

\newcommand{\bval}{$B_{val}$}
\newcommand{\bqrm}{$B_{qrm}$}
\newcommand{\bvot}{$B_{vot}$}
\newcommand{\bbal}{$B_{bal}$}
\newcommand{\bvalm}{B_{val}}
\newcommand{\bqrmm}{B_{qrm}}
\newcommand{\bvotm}{B_{vot}}
\newcommand{\bbalm}{B_{bal}}
\newcommand{\vp}{$v_{p}$}
\newcommand{\vval}{$v_{val}$}
\newcommand{\vbal}{$v_{bal}$}
\newcommand{\votesbeta}{$\text{\textit{Votes}}(\beta)$}
\newcommand{\maxvotep}{\textit{MaxVote(b, p, $\beta$)}}
\newcommand{\maxvoteq}{\textit{MaxVote(b, Q, $\beta$)}}
\DeclareMathAlphabet{\mathpzc}{OT1}{pzc}{m}{it}
\DeclareMathAlphabet{\mathcalligra}{T1}{calligra}{m}{n}

\documentclass[10 pt]{beamer}

\mode<presentation>
{
\usetheme{Warsaw}
\usecolortheme{seagull}

%\usecolortheme{dolphin}
%\usecolortheme{seahorse}
%\usecolortheme{albatross}
\setbeamercovered{transparent}
}


\usepackage[english]{babel}
\usepackage[utf8]{inputenc}

% font definitions, try \usepackage{ae} instead of the following
% three lines if you don't like this look

\usepackage{mathptmx}
%\usepackage[scaled=.90]{helvet}
\usepackage{courier}
\usepackage{chancery}
\usepackage{mathrsfs}
\usepackage{amsmath}
\usepackage{upgreek}


\usepackage[T1]{fontenc}


%\title{The Part-Time Parliament}
\title[{\makebox[.45\paperwidth]{The Part-Time Parliament\hfill\insertframenumber/\inserttotalframenumber}}]{The Part-Time Parliament} 
\subtitle{Leslie Lamport, 1998}

% - Use the \inst{?} command only if the authors have different
%   affiliation.
%\author{F.~Author\inst{1} \and S.~Another\inst{2}}
\author{Carlos Eduardo B. Bezerra}

% - Use the \inst command only if there are several affiliations.
% - Keep it simple, no one is interested in your street address.
\institute[Universities of]
{
Faculty of Informatics\\
Universittà della Svizzera italiana
}

\date{May 25th, 2011}


% This is only inserted into the PDF information catalog. Can be left
% out.
\subject{Talks}



% If you have a file called "university-logo-filename.xxx", where xxx
% is a graphic format that can be processed by latex or pdflatex,
% resp., then you can add a logo as follows:

% \pgfdeclareimage[height=0.5cm]{university-logo}{university-logo-filename}
% \logo{\pgfuseimage{university-logo}}



% % Delete this, if you do not want the table of contents to pop up at
% % the beginning of each subsection:
% %\AtBeginSubsection[]
% %{
% %\begin{frame}<beamer>
% %\frametitle{Outline}
% %\tableofcontents[currentsection,currentsubsection]
% % %\end{frame}
%}

% If you wish to uncover everything in a step-wise fashion, uncomment
% the following command:

%\beamerdefaultoverlayspecification{<+->}

\begin{document}



\begin{frame}
  \titlepage
\end{frame}



\begin{frame}
  \frametitle{Outline}
  \tableofcontents
  % You might wish to add the option [pausesections]
\end{frame}



%==========================================================
\section{Introduction}
%==========================================================

%----------------------------------------------------------
\subsection{A long way to Paxos}
%----------------------------------------------------------

\begin{frame}
  \frametitle{Introduction}

  Interesting information about `The Part-Time Parliament':
  
  \begin{itemize}
    \item the algorithm was first described in 1989 as a technical report [1]
    \item the paper was submitted in 1990\ldots and accepted in 1998%people just couldn't understand it!
    \item several other publications, such as [2, 3, 4], had explaining it as their main purpose\ldots
  \end{itemize}
  
  \vspace{3mm}
  
  \pause
  
  Some people do use the described \textbf{Paxos} protocol: % 20 years after being first described?...

  \begin{itemize}
    \item Google uses Paxos in their Chubby distributed lock service
    \item IBM supposedly uses Paxos in their SAN Volume Controller product
    \item Microsoft uses Paxos in the Autopilot cluster management service
    \item WANDisco uses Paxos in their replication technology
    \item Keyspace uses Paxos as its basic replication primitive
  \end{itemize}

\end{frame}



\begin{frame}

  The paper mostly uses an analogy with the fictional ancient island of Paxos, where:
  \begin{itemize}
    \item The parliament's (group's) legislators (processes) and their messengers (communication channels) may be unpredictably absent;
    \item There may be several ballots for each decree to be passed (rounds in each consensus instance);
    \item Legislators -- and messengers -- may be lazy or absent, but not dishonest (non-Byzantine failure model*)
  \end{itemize}
  \vspace{6 mm}
  {\small * The algorithm was extended later, however, to tolerate Byzantine failures [5]}
\end{frame}




%==========================================================
\subsection{The consensus problem}
%==========================================================


\begin{frame}
  \frametitle{Definition}
  
  \textbf{Consensus} is the process of agreeing on one single result among a group of participants. Assuming a collection of processes that can propose values, a consensus algorithm ensures that:
  
  \begin{itemize}
    \item only a single value among the proposed ones may be chosen;
    \item if no value has been proposed, then no value is chosen;
    \item a process never learns that a value has been chosen unless it really has been.
  \end{itemize}

%A process may be unaware of a given chosen value, as long as it doesn't learn an invalid one

%These properties guarantee consistency (safety), but not progress (liveness)
%Hard because of assuming the possibility of process failures\ldots

\end{frame}



%==========================================================
\section{About correctness}
%==========================================================

%----------------------------------------------------------
\subsection{Notations used}
%----------------------------------------------------------


\begin{frame}
%  \frametitle{The Paxos algorithm}
  \frametitle{Some notations used}
  
  A value is chosen after a series of numbered \textbf{ballots}. For each ballot $B$, we have that:
  \begin{itemize}
    \item \bbal\ is its ballot number; %loosely related to its order
    \item \bqrm\ is a nonempty finite set of processes (quorum);
    \item \bvot\ is the set of processes who voted in $B$ -- each process may decide to participate (vote) or not in $B$;
    \item \bval\ is the value being voted on in that ballot.
  \end{itemize}
  
  \vspace{4 mm}
  A ballot is said to be \emph{successful} when \bqrm\ $\subseteq$ \bvot\ -- which means that every quorum member voted.

\end{frame}

\begin{frame}
%  \frametitle{The Paxos algorithm}
  \frametitle{Some more notations}
  
  In each ballot, each process can cast a vote (accept or not a given proposed value). A vote $v$ has the components:

  \begin{itemize}
    \item \vp, which is the process $p$ who cast it;
    \item \vbal, the number of the ballot for which it was cast;
    \item \vval, a value voted on.
  \end{itemize}
  
  \vspace{4 mm}
 % A \emph{null} vote $v$ is that for which \vbal\ $=$ $-\infty$ and \vval\ $=$ BLANK. 
 For the set $\beta$ of ballots, the set \votesbeta\ is defined as\\
 $\{v$ : $\exists B$ $\in$ $\beta$ | \vp\ $\in$ \bvot\ $\wedge$ \vbal\ $=$ \bbal\ $\wedge$ \vval\ $=$ \bval\ $\}$ 

\end{frame}

\begin{frame}
%  \frametitle{The Paxos algorithm}
  \frametitle{Still more notations\ldots}

  We then define \maxvotep\ as the vote $v$ with highest \vbal\ in 
 $\{v \in Votes(\beta) : (v_p = p) \wedge (v_{bal} < b)\} \cup \{null\}$,\\
 where $null$ means that $p$ didn't vote in any $B \in \beta$ ($null_{bal} = -\infty$)

  \vspace{4 mm}
  We also define \maxvoteq\ as the vote $v$, such that \vbal\ is the highest among all \maxvotep\ with $p \in Q$

  \vspace{8 mm}
  {\small With such notations, we can specify the conditions which ensure consistency and allow for progress.}

\end{frame}

%----------------------------------------------------------
\subsection{Consistency conditions}
%----------------------------------------------------------


\begin{frame}
  \frametitle{Consistency conditions}

  Consistency is guaranteed if the following conditions are ensured:

%write conditions on the whiteboard
  \begin{itemize}
    \item $B1(\beta)$
    
    $\forall B, B' \in \beta : (B \neq B') \Rightarrow (\bbalm \neq B'_{bal})$
   
    \pause
    \vspace{4 mm}
    \item $B2(\beta)$ %B2 may be ensured by always picking a majority of processes as quorum, which implies that n > 2f (a majority is correct for each instance)
  
    $\forall B, B' \in \beta : \bqrmm \cap B'_{qrm} \neq \emptyset)$
    
    \pause
    \vspace{4 mm}
    \item $B3(\beta)$
  
    $\forall B, B' \in \beta :$\\
    $(\text{\textit{MaxVote}}(\bbalm, \bqrmm, \beta)_{bal} \neq -\infty) \Rightarrow$\\
    $\Rightarrow (\bvalm = \text{\textit{MaxVote}}(\bbalm, \bqrmm, \beta)_{val})$
  \end{itemize}

\end{frame}



%----------------------------------------------------------
\subsection{Proof sketch}
%----------------------------------------------------------


\begin{frame}
\frametitle{Proof sketch}

%write lemma on the whiteboard
\textsc{\textbf{Lemma}}. If $B1(\beta)$, $B2(\beta)$ and $B3(\beta)$ hold, then $$((\bqrmm \subseteq \bvotm) \wedge (B'_{bal} > \bbalm)) \Rightarrow (B'_{val} = \bvalm)$$ for any $B$, $B'$ in $\beta$.

\pause
\vspace{3 mm}
\textsc{\textbf{Proof of Lemma}}. Let $\Psi(B, \beta)$ be defined as: $$\Psi(B, \beta) = \{B' \in \beta : (B'_{bal} > \bbalm) \wedge (B'_{val} \neq \bvalm)\}$$


Assume that $\bqrmm \subseteq \bvotm$. By contradiction, assume that $\Psi(B, \beta) \neq \emptyset$.

\pause
\vspace{3 mm}
(1) As $\Psi(B, \beta) \neq \emptyset$, we pick $C \in \Psi(B, \beta)$, such that $C_{bal} = min\{B'_{bal} : B' \in \Psi(B, \beta)\}$

\pause
\vspace{2 mm}
(2) From (1) and from the definition of $\Psi(B, \beta)$, we have that $C_{bal} > \bbalm$

\end{frame}

\begin{frame}
\frametitle{Proof sketch}
\framesubtitle{continuing\ldots}

(3) As $\bqrmm \subseteq \bvotm$ and $\forall B, B' \in \beta : \bqrmm \cap B'_{qrm} \neq \emptyset$, we have that $\bvotm \cap C_{qrm} \neq \emptyset$

\pause
\vspace{2 mm}
(4) As $C_{bal} > \bbalm$ and $\bvotm \cap C_{qrm} \neq \emptyset$, we have that \textit{MaxVote}$(C_{bal}, C_{qrm}, \beta)_{bal} \geq \bbalm$

\pause
\vspace{2 mm}
(5) From (4) and the definition of \textit{MaxVote}$(C_{bal}, C_{qrm}, \beta)$, we know that such vote exists in \textit{Votes}$(\beta)$ (i.e. it is not $null$).

\pause
\vspace{2 mm}
(6) From (5) and $B3(\beta)$, we have that \textit{MaxVote}$(C_{bal}, C_{qrm}, \beta)_{val} = C_{val}$

\pause
\vspace{2 mm}
(7) From (6), and as $C_{val} \neq \bvalm$ (from $\Psi(B, \beta$)), we have that \textit{MaxVote}$(C_{bal}, C_{qrm}, \beta)_{val} \neq \bvalm$

\end{frame}



\begin{frame}
\frametitle{Proof sketch}
\framesubtitle{continuing\ldots}

(8) As \textit{MaxVote}$(C_{bal}, C_{qrm}, \beta)_{bal} \geq \bbalm$ (from (4)) and \textit{MaxVote}$(C_{bal}, C_{qrm}, \beta)_{val} \neq \bvalm$ (from (7)), we have that \textit{MaxVote}$(C_{bal}, C_{qrm}, \beta)_{bal} > \bbalm$

\pause
\vspace{2 mm}
(9) From (8), and since \textit{MaxVote}$(C_{bal}, C_{qrm}, \beta)_{val} \neq \bvalm$, we have that \textit{MaxVote}$(C_{bal}, C_{qrm}, \beta) \in \text{\textit{Votes}}(\Psi(B,\beta))$ 

\pause
\vspace{2 mm}
(10) By the definition of \textit{MaxVote}$(C_{bal}, C_{qrm}, \beta)$, we have that \textit{MaxVote}$(C_{bal}, C_{qrm}, \beta)_{bal} < C_{bal}$

\pause
\vspace{2 mm}
(11) From (10), we have that $\exists B' : B' \in \Psi(B, \beta) \wedge B'_{bal} < C_{bal}$, which means that $$C_{bal} \neq min\{B'_{bal} : B' \in \Psi(B, \beta)\}$$ which is a contradiction with (1). $\square$

\end{frame}



\begin{frame}
\frametitle{Proof sketch}
\framesubtitle{continuing\ldots}

With the lemma, we show that, for a given set $\beta$ of ballots which hold $B1$, $B2$ and $B3$, then any two successful ballots decided the same value.

\vspace{2 mm}
\textbf{\textsc{Theorem 1}}. If $B1(\beta)$, $B2(\beta)$ and $B3(\beta)$ hold, then
$$((\bqrmm \subseteq \bvotm) \wedge (B'_{qrm} \subseteq B'_{vot})) \Rightarrow (B'_{val} = B_{val})$$
for any $B$, $B'$ in $\beta$.

\pause
\vspace{2 mm}
\textbf{\textsc{Proof of Theorem}}. If $B'_{bal} = B_{bal}$ then $B1(\beta)$ implies $B' = B$. If $B'_{bal} \neq B_{bal}$, the theorem follows immediately from the lemma. $\square$

\end{frame}



\begin{frame}
\frametitle{Proof sketch}
\framesubtitle{continuing\ldots}

If there are enough correct (non-faulty) processes, then a new ballot may be conducted while $B1$, $B2$ and $B3$ are preserved.

\vspace{2 mm}
\textbf{\textsc{Theorem 2}}. Let $b$ be a ballot number, and let $Q$ be a set of processes such that $b > \bbalm$ and $Q \cap \bqrmm \neq \emptyset$, for all $B \in \beta$. If $B1(\beta)$, $B2(\beta)$ and $B3(\beta)$ hold, then there is a ballot $B'$ with $B'_{bal} = b$ and $B'_{qrm} = B'_{vot} = Q$ such that $B1(\beta \cup \{B'\})$, $B2(\beta \cup \{B'\})$ and $B3(\beta \cup \{B'\})$ hold.

\pause
\vspace{2 mm}
\textbf{\textsc{Proof of Theorem}}.
\begin{itemize}
\item $B1(\beta \cup \{B'\})$ follows from $B1(\beta)$ and the fact that $B'_{bal} = b$, and the assumption about $b$.
\item $B2(\beta \cup \{B'\})$ follows from $B2(\beta)$ and the fact that $B'_{qrm} = Q$, and the assumption about $Q$.
\item For $B3(\beta \cup \{B'\})$, if \textit{MaxVote}$(b,Q,B) = -\infty$ then let $B'_{val}$ be any value; otherwise, let $B'_{val} = \text{\textit{MaxVote}}(b,Q,B)_{val}$. $B3(\beta \cup \{B'\})$ then follows from $B3(\beta)$. $\square$
\end{itemize} 
\end{frame}



%==========================================================
\section{The Synod Protocol}
%==========================================================

%----------------------------------------------------------
\subsection{Preliminary protocol}
%----------------------------------------------------------


\begin{frame}
\frametitle{Preliminary protocol}

A protocol can be derived directly from $B1(\beta)$, $B2(\beta)$ and $B3(\beta)$, so that all these properties are maintained throughout its execution and a value may* be agreed upon.

\begin{itemize}
  \item To maintain $B1(\beta)$, whenever $p$ starts a new ballot $B$, it assigns a sequence number $seq$, which was never used by $p$. To avoid collisions, $p$'s unique id is appended. Then $\bbalm = (seq, p_{id})$.
  
  \pause
  \item To maintain $B2(\beta)$, each ballot $B$ has \bqrm\ as the majority set of the processes. As it is impossible to choose two disjoint majority sets from the same group*, $B2(\beta)$ is satisfied.
  
  \pause
  \item To maintain $B3(\beta)$, before initiating a ballot $B$, $p$ has to find out what is \textit{MaxVote}$(B, \bqrmm, \beta)$, which is done by finding the \textit{MaxVote}$(B, q, \beta)$ for each $q$ in \bqrm.
  \begin{itemize}
    \item If \textit{MaxVote}$(B, \bqrmm, \beta) = null$, then \bval\ can be anything;\\
    otherwise $\bvalm = \text{\textit{MaxVote}}(B, \bqrmm, \beta)_{val}$.
  \end{itemize}
\end{itemize}

\pause
{\small * If half of the processes is faulty, consistency is still respected and there is no deadlock, but the ballot $B$ does not succeed}

\end{frame} 



\begin{frame}
\frametitle{Preliminary protocol}%all the phases are optional! consistency is always guaranteed!

  Phase 1) Find \textit{MaxVote}$(B, \bqrmm, \beta)_{val}$
   
  \pause
  \hspace{3 mm} \textbf{Phase 1a}) $p$ picks a ballot number $b$ and sends \textit{NextBallot}$(b)$ to other processes
  
  \pause 
  \hspace{3 mm} \textbf{Phase 1b}) each process $q$ responds with \textit{LastVote}$(b, v)$ to $p$, where $v$ is the vote cast by $q$ with the highest ballot number less than $b$ (or $null$)
   
  \vspace{6 mm}
  
  \pause 
  Phase 2) Execute the ballot $B$
  
  \pause 
  \hspace{3 mm} \textbf{Phase 2a}) after receiving \textit{LastVote}$(b, v)$ from a majority $Q$ of the processes, $p$ initiates the ballot with number $b$ and value $val$ such that $B3(\beta)$ is maintained, by sending \textit{BeginBallot}$(b, val)$ to every process in $Q$%Bqrm = Q, Bval = val, Bbal = b
  
  \pause    
  \hspace{3 mm} \textbf{Phase 2b}) upon receiving \textit{BeginBallot}$(b, val)$, $q$ will not vote on it if it has already sent \textit{LastVote}$(b', v')$, such that $b' > b$; otherwise, it sends \textit{Voted}$(b, q)$ to $p$ and records this vote in its memory
  
\end{frame}



\begin{frame}
\frametitle{Preliminary protocol}
   
  Phase 3) Announce the result of consensus
  
  \pause 
  \hspace{3 mm} \textbf{Phase 3a}) if $p$ receives \textit{Voted}$(b, q)$ from a majority $Q$ of processes, then it sends \textit{Success}$(val)$ to all processes
  
  \pause    
  \hspace{3 mm} \textbf{Phase 3b}) upon receiving \textit{Success}$(val)$, the process learns that the group achieved consensus on $val$.
  
  \pause
  \vspace{6mm}
  Every phase of this protocol is optional. Consistency is still maintained, although progress is not guaranteed.

  \vspace{6mm}
  Each process $q$ must keep \textit{MaxVote(b, q, $\beta$)} for each ballot numbered $b$ and remember all \textit{LastVote}$(b, v)$ messages it has sent.
 
\end{frame}



%----------------------------------------------------------
\subsection{Basic protocol}
%----------------------------------------------------------


\begin{frame}
\frametitle{Basic protocol}

In the preliminary protocol, much information must be kept. A simpler, yet correct, protocol can be derived. Each process $p$ keeps then only:

\vspace{2 mm}
\mbox{\textit{\textbf{lastTried}}$[p]$ -- the number of the last ballot $p$ tried to initiate ($-\infty$ if none)}

\vspace{2mm}
\mbox{\textit{\textbf{prevVote}}$[p]$ -- the highest ballot for which $p$ ever voted ($-\infty$ if none)}

\vspace{2mm}
\textit{\textbf{nextBal}}$[p]$ -- the highest ballot number $b$ for which $p$ sent \textit{LastVote}$(b, v)$ ($-\infty$ if none)

%\vspace{4mm}
%In the preliminary protocol, \textit{LastVote}$(b, v)$ represented a ``promise'' from $q$ not to vote in any ballot numbered $b' : \text{\textit{LastVote}}(b, v)_{bal} < b' < b$ 

\vspace{4mm}
%Here, 
\textit{LastVote}$(b, v)$ represents a ``promise'' from $q$ not to vote in any ballot numbered $b' < b$
% although this promise is stronger, it just representing not sending a LastVote(...) in cases the preliminary prot. would. 
% as seen, this doesn't affect consistency
% ANYway, this is arguable, as one could see that the collection of promises from the preliminary are equivalent to this
% one single promise.

\end{frame}



\begin{frame}
\frametitle{Basic protocol}

  Phase 1)
   
  \hspace{3 mm} \textbf{Phase 1a}) $p$ picks a ballot number $b > \text{\textit{lastTried}}[p]$ and sends \textit{NextBallot}$(b)$ to other processes
   
   %explain the if thing.
  \hspace{3 mm} \textbf{Phase 1b}) upon receipt of \textit{NextBallot}$(b)$ from $p$, \mbox{\textbf{if \textit{b}} > \textbf{\textit{nextBal}}[\textbf{\textit{q}}],} $q$ \mbox{sets $\text{\textit{nextBal}}[q]$ to $b$ and sends \textit{LastVote}$(b, v)$ to $p$, where $v = \text{\textit{prevVote}}[q]$}
   
  \vspace{6 mm}
  
  \pause  
  Phase 2)
   
  \hspace{3 mm} \textbf{Phase 2a}) after receiving \textit{LastVote}$(b, v)$ from a majority $Q$, \textbf{where \mbox{\textit{b $=$ lastTried}}}[\textbf{\textit{p}}], $p$ initiates ballot  number $b$ with value $val$ such that $B3(\beta)$ is maintained, by sending \textit{BeginBallot}$(b, val)$ to every process in $Q$
  %Bqrm = Q, Bval = val, Bbal = b
      
  \hspace{3 mm} \textbf{Phase 2b}) upon receiving \textit{BeginBallot}$(b, val)$ \textbf{with \textit{b $=$ nextBal}}[\textbf{\textit{q}}], $q$ votes on ballot number $b$, set $prevVote[q]$ to this vote and sends \textit{Voted}$(b, q)$ to $p$
  % here, each process first must promise to vote for the ballot B (and not receive any higher ballot B') to be able to vote for B...
  % in the preliminary, it didn't need any promise to respond in phase 2b
  
\end{frame}



\begin{frame}
\frametitle{Basic protocol}
   
  Phase 3)
   
  \hspace{3 mm} \textbf{Phase 3a}) if $p$ receives \textit{Voted}$(b, q)$ from a majority $Q$ of processes, then it sends \textit{Success}$(val)$ to all processes
      
  \hspace{3 mm} \textbf{Phase 3b}) upon receiving \textit{Success}$(val)$, the process learns that the group achieved consensus on $val$.

  \pause  
  \vspace{6mm}
  Consistency is still maintained, and progress is still not guaranteed.

  \vspace{6mm}
  Anyway, it is impossible to guarantee termination of consensus in an asynchronous system [6]\ldots 
 
\end{frame}



%----------------------------------------------------------
\subsection{Complete protocol}
%----------------------------------------------------------


\begin{frame}
  \frametitle{Complete protocol}
  
  \begin {itemize}
    \item Same steps of the basic protocol  
    
    \vspace{2mm}
    \item Each process performs phases 1b -- 3b as soon as they can  
    
    \vspace{2mm}
    \item Some process must execute phase 1a
    \begin{itemize}
      \item Never initiating a ballot prevents termination
      \item However, too frequent initiation also does it %explicar cada ballot anulando o anterior, e sendo anulado antes de terminar
      \item A single process (leader) should be the only one initiating ballots %isso tb causaria a anulação de ballots anteriores de outros processos
    \end{itemize}
    
    \vspace{2mm}
    \item Knowledge of time, and of a maximum message delay $\Delta$, is needed
    \begin{itemize}
      \item Not exactly an asynchronous system anymore
    \end{itemize}
  \end{itemize}
  
\end{frame}



\begin{frame}
  \frametitle{Complete protocol}
  
  \begin {itemize}
    \item A minimal execution of the basic protocol would take at most $5\Delta$
%     \begin {itemize}
%       \item the responses to message (1a) would arrive after $2\Delta$ (at most)
%       \item the responses to message (2a) would arrive after $2\Delta$ (at most)
%       \item the (3a) message would take at most $\Delta$
%     \end{itemize} 
    \pause
    \vspace{2 mm}
    \item A leader would ensure progress, but that requires leader election     
    \begin {itemize}
      \item when a leader seems to have failed, a new one should be elected\\
      (the election takes time $T_{el}$)
      \item if not enough responses to (1a) or (2a) arrive after $2\Delta$, a ballot $B$ had started before $T_{el}$
      \begin{itemize}
        \item or too many failures or message losses occurred\ldots %which violates the assumptions for this protocol
      \end{itemize}
      %the other ballot may even have succeeded! 
      %maybe started by another process before T was fully elapsed
      \item in this case, a ballot with number $b' > B_{bal}$ may be started by the leader -- $NACK$s to (1a) or (2a) may have carried \bbal
      \item in the worst case, the leader receives a $NACK$ for (2a), $4\Delta$ after $T_{el}$
      \item as no other process initiated any ballot after $T_{el}$, the ballot numbered $b'$ would succeed $5\Delta$ after the $NACK$
    \end{itemize}
    \vspace{2 mm}
    \pause
    \item The complete protocol, with leader election, terminates at most after $T_{el} + 9\Delta$
    \begin{itemize}
      \item if a majority of processes didnt't fail during that time
      \item and if the leader didn't fail for $9\Delta$ after $T_{el}$
      \item and if enough messages were delivered
    \end{itemize}   

  \end{itemize}
  
\end{frame}



%==========================================================
\section{The Paxos Protocol}
%==========================================================

%----------------------------------------------------------
\subsection{Running multiple consensus instances}
%----------------------------------------------------------


\begin{frame}
  \frametitle{The \textbf{Paxos} Protocol}
  \framesubtitle{Multiple consensus instances}
  
  \begin{itemize}
    \item Each instance has an id $I_{id}$ -- and there should be a leader %although, as seen, that wouldn't be really necessary for consistency

    \pause
    \item Phases (1a) and (1b) can be executed for many of them --
    e.g. $p$ sends $NextBallot(b,n)$, for all instances $I$ with $I_{id} > n$
    \item The response from $q$ (1b) would contain a $LastVote(I_{id}, b, v)$ message for each instance $I : I_{id} > n$ that $q$ voted in
    \item Another process can send $NextBallot(b',n')$, with $b' > b$,\\
    and $p$'s (2a) messages would be ignored from then on
    \item This brings the protocol latency, on most cases, down to $3\Delta$
    \begin{itemize}
      \item no leader changes, enough correct processes and enough messages
    \end{itemize}
    
    \pause
    \item If each process sent messages (2b) to every other one, we'd have $2\Delta$ %show how this is done on the board
        
    \pause
    \item If instance $I^A$ was passed (phase (3b) concluded) before $I^B$ had a value proposed (phase (2a)), then $I^A_{id} < I^B_{id}$
    % ORDERING PROPERTY
    % the leader needed to know the result of previous instances before starting a new one (= atomic broadcast)
  \end{itemize}

\end{frame}



%----------------------------------------------------------
\subsection{Relevance to computer science}
%----------------------------------------------------------


\begin{frame}
  \frametitle{The \textbf{Paxos} Protocol}
  \framesubtitle{Relevance to computer science}
  
  \begin{itemize}
    \item It may be used to implement state-machine replication
    \begin{itemize}
      \item A system (e.g. a database) may be implemented as a set of states and deterministic state changes
      \item The group of processes would be the set of replicas of the system
      \item Each state changed would be agreed upon by means of consensus (an ordered instance of Paxos)
      \item One of the replicas might be elected as leader, and client requests (state changes) would be sent to it
      \item As all replicas would have a consistent set of state changes, so would be their state
    \end{itemize} 
  \end{itemize}
  
  \vspace{2 mm}
  \begin{itemize}
    \item Many derivations from Paxos have been published since it was first described
    \begin{itemize}
      \item Cheap Paxos, Fast Paxos, Generalized Paxos, Multi-Paxos,\\
      Byzantine Paxos, Fast Byzantine Paxos, Fast Byzantine Multi-Paxos, Ring Paxos
    \end{itemize} 
  \end{itemize}

\end{frame}



%==========================================================
\section*{References}
%==========================================================


\begin{frame}
  \frametitle{References}

[1] Lamport, L. \textbf{The part-time parliament}, Technical Report 49, Systems Research Center, Digital Equipment Corp., 1989

\vspace{4 mm}
[2] Lamport, L. \textbf{Paxos made simple}, ACM SIGACT News, 2001

\vspace{4 mm}
[3] De Prisco, R., Lampson, B., Lynch, N. \textbf{Revisiting the Paxos algorithm}, Distributed Algorithms, 1997

\vspace{4 mm}
[4] Lampson, B. \textbf{How to build a highly available system using consensus}, Distributed Algorithms, 1996

\vspace{4 mm}
[5] Castro, M., Liskov, B., \textbf{Practical Byzantine fault tolerance}, Operating Systems Design and Implementation, 1999

\vspace{4 mm}
[6] Fischer, M., Lynch, N., Paterson, M., \textbf{Impossibility of distributed consensus with one faulty process}, J. ACM 32 1 (Jan.), 374-382, 1985
  
\end{frame}



\begin{frame}

\begin{center}
{\Huge \textbf{Thank you!}}

\vspace{5 mm}
{\footnotesize Questions?}
\end{center}
% 
%   \begin{itemize}
%     \item The paper presents the construction of a consensus protocol, based on three consistency conditions, from a preliminary one, to a complete one which ensures consistency \textbf{and} termination, as long as some message delay bound is known
%   \end{itemize}
% % 
% % The following outlook is optional.
% \vskip0pt plus.5fill
% \begin{itemize}
%   \item Outlook
%   \begin{itemize}
%     \item Something you haven't solved.
%     \item Something else you haven't solved.
%   \end{itemize}
% \end{itemize}

\end{frame}



\end{document}
